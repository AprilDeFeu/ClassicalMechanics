\documentclass{article}

\usepackage{NotesStyle}

\title{Classical Mechanics}
\author{April Sada Solomon}

\begin{document}
	
	% Title page
	\clearpage
	\maketitle
	\newpage
	
	% Table of Contents.
	\tableofcontents
	\newpage
	
	% Page numbering [Main content]
	\setcounter{page}{1}
	\cfoot{\thepage}

	
	\part{Newtonian Mechanics}
		\section{Fundamentals}
		The basis of our understanding classical mechanics relies on approximation and simplification of observed phenomena relating to human events. This means that classical mechanics deals with assumptions on physical events that are occur at relatively normal speeds over relatively normal distances. Hence, once these simplifications are thoroughly understood, it becomes easier to focus on details that modify them and broaden our understanding of physics at extreme speeds and distances. It is therefore necessary to establish a set of fundamental facts and definitions we may use throughout.
		
		\begin{defn}
			The set of all real numbers is denoted $\R$.
		\end{defn}
		\begin{defn}
			An $n-$dimensional \textbf{real vector space} $\R^n$, whose elements are called \textbf{vectors}, is defined as an algebraic modular structure with the following axioms:
			\begin{table}[h]
				\begin{tabular}{|c|c|}
					\hline
					\textbf{Axiom} & \textbf{Implication} \\ \hline
					Associative law of vector addition & $\forall \, \vec{u}, \vec{v}, \vec{w} \in \R^n: \quad \vec{u} + \left(\vec{u} + \vec{w}\right) = \left(\vec{u} + \vec{v}\right) + \vec{w}$ \\ \hline
					Commutative law of vector addition & $\forall \, \vec{u}, \vec{v} \in \R^n: \quad \vec{u} + \vec{v} = \vec{v} + \vec{u}$ \\ \hline
					Identity element of vector addition & $\exists\, \vec{0}, \in \R^n, \,\, \forall \, \vec{v} \in \R^n: \quad \vec{0} + \vec{v} = \vec{v}$ \\ \hline
					Inverse elements of vector addition & $\forall \, \vec{v} \in \R^n, \,\, \exists\, -\vec{v} \in \R^n: \quad \vec{v} + \left(-\vec{v}\right) = \vec{0}$ \\ \hline
					Associative law of scalar multiplication & $\forall \,a, b \in R, \,\, \forall\, \vec{v} \in \R^n: \quad a\left(b\vec{v}\right) = \left(ab\right) \vec{v}$ \\ \hline
					Identity element of scalar multiplication & $\exists \,1 \in \R, \,\,\forall\, \vec{v} \in \R^n:\quad 1\vec{v} = \vec{v}$                     \\ \hline
					First distributive law & $\forall \, a \in \R, \,\, \forall \,\vec{v}, \vec{u} \in \R^n: \quad a\left(\vec{u} + \vec{v}\right) = a\vec{u} + a\vec{v}$ \\ \hline
					Second distributive law & $\forall \, a, b \in R, \,\, \forall \, \vec{v} \in \R^n: \quad \left(a+b\right) \vec{v} = a\vec{v} + b\vec{v}$\\ \hline
				\end{tabular}
			\end{table}
		\end{defn}
		\begin{defn}
			An $n-$dimensional \textbf{affine space} $A^n$, whose elements are called \textbf{points}, is a geometric structure associated with the vector space $\R^n$ by the linear map $A^n \times \R^n \to A^n: \,\, \left(a,\vec{v}\right) \mapsto a + \vec{v}$ with the following properties:
			\begin{table}[h]
				\begin{tabular}{|c|c|}
					\hline
					\textbf{Property} & \textbf{Implication} \\ \hline
					Right identity & $\forall \, a \in A^n, \,\, \vec{0} \in \R^n: \quad a + \vec{0} = a$ \\ \hline
					Associativity & $\forall \, \vec{v}, \vec{w} \in \R^n, \,\, \forall \, a \in A^n: \quad \left(a+\vec{v} \right) + \vec{w} = a + \left(\vec{v} + \vec{w} \right)$ \\ \hline
					Free, transitive action &  $\forall \, a \in A^n, \,\, \forall \vec{v} \in \R^n: \quad \R^n \to A^n: \, \vec{v} \mapsto a + \vec{v} $ is bijective \\ \hline
					Injective translations & $\forall \, \vec{v} \in \R^n, \,\, \forall a \in A^n: \quad A^n \to A^n: \, a \mapsto a + \vec{v}$ is bijective. \\ \hline
				\end{tabular}
			\end{table}			
		\end{defn}
		Affine spaces differ from vector spaces particularly in that they do not have a preferred origin. This implies that for any two arbitrary points $a, b \in A^n$, there is no defined sum. Instead, their difference defines a vector in $\R^n$.
		\begin{defn}
			A \textbf{vector} $v \in \R^n$ is given by the difference between two points $a, b \in A^n$ such that
			\begin{equation}
				\label{eq:VectorFromAffine}
				\vec{v} = b - a
			\end{equation}
		\end{defn}
		\begin{defn}
			The \textbf{inner product space} is defined as a vector space $\R^n$ such that the map $\left\langle \cdot, \cdot\right\rangle \, : \, \R^n \times \R^n \to \R$ satisfies the following three properties $\forall\,  \vec{u}, \vec{v}, \vec{w} \in \R^n$ and $\forall \, a, b, \in \R$:
			\begin{table}[h]
				\begin{tabular}{|c|c|}
					\hline
					\textbf{Property} & \textbf{Implication} \\ \hline
					Symmetry & $\left\langle \vec{u}, \vec{v} \right\rangle = \left\langle \vec{v}, \vec{u} \right\rangle$ \\ \hline
					Linearity in first element & $\left\langle a\vec{u} + b\vec{v}, \vec{w} \right\rangle = a \left\langle \vec{u}, \vec{w} \right\rangle + b\left\langle \vec{v}, \vec{w} \right\rangle$ \\ \hline
					Positive-definite & $ \vec{u} \neq \vec{0}\implies \left\langle \vec{u}, \vec{u} \right\rangle > 0 $ \\ \hline
				\end{tabular}
			\end{table}
		\end{defn}
		\begin{defn}
			The \textbf{distance} between two points $a,b$ in some affine space $A^n$ is defined as
			\begin{equation}
				\label{eq:EuclidDistance}
				\rho\left(a,b\right) = \left|\left| a-b \right|\right| = \sqrt{\sum_{i=1}^{n} \left(a_i - b_i\right) \cdot \left(a_i - b_i\right) }
			\end{equation}
			And label the affine space as a \textbf{Euclidean space}.
		\end{defn}
		\begin{defn}
			A \textbf{coordinate system} in $\R^n$ is given by identifying a specific origin and some basis that serves to indicate used axes and their direction, as well as how unit vectors are defined within $\R^n$.
		\end{defn}
		
		\subsection{Galileo's Relativity Principle}
		Galileo asserted that there exists \textbf{inertial coordinate systems} with the following properties:
		\begin{itemize}
			\item The laws of physics at any point in time will remain the same.
			\item All coordinate systems in uniform rectilinear motion with respect to an inertial coordinate system are themselves inertial as well.
		\end{itemize}
		To understand this, assume we define an inertial coordinate system somewhere on the surface of the Earth. Then, assume the room we are in has no windows making it impossible to see outside. Would we be able to make experiments, such as juggling illegal Schedule I narcotics, to determine whether the room is moving in a straight line at constant speed, or staying at rest? The answer is no. Why? Legally, said narcotics are illegal. Physically, let us elaborate using a more refined framework.
		
		\begin{defn}
			A \textbf{Galilean space-time structure} or \textbf{Galilean space} is given by the following three elements:
			\begin{enumerate}
				\item The \textbf{universe} is said to be an affine space $A^4$ whose elements are called \textbf{world-points} or \textbf{events}, and is directly associated with the vector space $\R^4$.
				\item \textbf{Time} is a linear mapping $t: \R^4 \to \R$, with \textbf{time intervals} given between two events $a, b \in A^4$ as $t\left(b-a\right)$. If $t(b-a) =0$ then $a,b$ are \textbf{simultaneous events}.
				\item The \textbf{distance} between two simultaneous events is given by an inner product on the linear subspace $\R^3 \subset \R^4$, defined as the kernel for linear mapping $t$ for two events $a, b \in A^3$ as given by equation (\ref{eq:EuclidDistance}). 
			\end{enumerate}
		\end{defn}
		
		\begin{defn}
			The \textbf{Galilean group} is the group of all transformations, called \textbf{Galilean transformations}, of a Galilean space which preserve its structure, meaning they preserve intervals of time and distance between simultaneous events.
		\end{defn}
		\begin{defn}
			A \textbf{Galilean coordinate space} is the direct product $(t, \vec{x}) \in \R \times \R^3: \,\,t \in \R, \vec{x} \in \R^3$, where $\R^3$ has a fixed inner product.
		\end{defn}
		We now provide examples of Galilean transformations on Galilean coordinate spaces.
		\begin{exmp}
			\textit{Uniform motion with velocity $\vec{v}$}
			$$ g_{1} \left(t, \vec{x}\right) = \left(t, \vec{x} + \vec{v}t \right) \quad \forall \, t \in \R, \, \forall \, \vec{x} \in \R^3$$
		\end{exmp}
		\begin{exmp}
			\textit{Translation of the origin by $(a, \vec{u})$}
			$$ g_2 \left(t, \vec{x}\right) = \left(t+a, \,\vec{x} + \vec{u}\right), \quad \forall \, t \in \R, \, \forall \, \vec{x} \in \R^3$$
		\end{exmp}
		\begin{exmp}
			\textit{Rotation of coordinate axes for orthogonal transform $G: \R^3 \to \R^3$}
			$$ g_3 \left( t, \vec{x} \right) = \left(t,\, G \vec{x} \right), \quad \forall \, t \in \R, \forall \, \vec{x} \in \R^3$$ 
		\end{exmp}
		\begin{thm}
			All Galilean transformations $g: \R \times \R^3 \to \R \times \R^3$ can be uniquely written as a composition of a rotation $g_3$, translation $g_2$, and uniform motion $g_1$ transformation such that $g = g_1 \circ g_2 \circ g_3$.
		\end{thm}
		\begin{proof}
			Let us define a Galilean transform $g: \, \R \times \R^3 \to \R \times \R^3 $ such that $R \in O\left(3\right), \tau \in \R,$ and $v, y \in \R^3$.
			The most general $\R$-affine map we can define is
			\begin{align*}
				g \left(t, \vec{x} \right) &=  A\left(t, \vec{x} \right) + \left( \tau, \vec{y}\right)\\
				&= \left( A_{11} t + A_{12} \vec{x}, A_{21} t + A_{22} \vec{x} \right) 
			\end{align*}
			In order for distance to be preserved, we know that $A_{22}$ must be an orthogonal matrix. We must also make sure that the time intervals are equally preserved. Assume we thus have two events $\vec{x}_1, \vec{x}_2$ such that
			$$ A_{11} \left( t_2 - t_1\right) + A_{12} \left(\vec{x}_2 - \vec{x}_1 \right) = t_2 - t_1$$
			We thus draw the implication that for all $\vec{x}_1, \vec{x}_2$ to hold, we need $A_{12} = 0$ which directly implies $A_{11} = 1$. Finally, the last component is the velocity defined with the linear map $\R \to \R^3$, showing that indeed, 
			$$ g : \, 
			\begin{pmatrix}
				t \\ \vec{x}
			\end{pmatrix} \mapsto 
			\begin{pmatrix}
				1 & 0 \\
				\vec{v} & R
			\end{pmatrix} 
			\begin{pmatrix}
				t \\ \vec{x}
			\end{pmatrix} +
			\begin{pmatrix}
				\tau \\ \vec{y}
			\end{pmatrix}
			$$
			Furthermore, we note the rotation group in $\R^3$ has 3 dimensions in itself, uniform motion includes a boost and spatial translation so $2\cdot 3 = 6$ dimensions, and lastly time translation is $1$ dimensional, which implies that the Galilean transformation group is 10-dimensional.
			
		\end{proof}
		\begin{defn}
			A \textbf{Galilean coordinate system on a set $M$} is an injective linear map $\varphi: \, M \to \R \times \R^3$. 
		\end{defn}
		\begin{defn}
			We say that $\varphi_2$ experiences \textbf{uniform motion} with respect to $\varphi_1$ if the linear map $$\varphi_1 \circ \varphi_2^{-1}: \, \R\times \R^3\to \R \times \R^3$$ is a Galilean transform.  
		\end{defn} 
		\begin{defn}
			A \textbf{motion} is a differentiable mapping $\vec{x}: \, I \to \R^n$ for some real interval $I$. 
		\end{defn}
		\begin{defn}
			The \textbf{velocity vector $\vec{v} \in \R^n$} is given by the first derivative with respect to time of a motion $\vec{x}$ such that
			\begin{equation}
				\label{eq:VelocityVector}
				\vec{v} \left(t_0 \right) = \dot{\vec{x}} \left( t_0 \right) = \left. \frac{d\vec{x}}{dt} \right|_{t = t_0} = \lim_{h\to 0} \frac{\vec{x} \left(t_0 + h\right) - \vec{x} \left( t_0\right)}{h}
			\end{equation}
		\end{defn}
		\begin{defn}
			The \textbf{acceleration vector $\vec{a} \in \R^n$} is given by the second derivative with respect to time of a motion $\vec{x}$ such that
			\begin{equation}
				\label{eq:AccelerationVector}
				\vec{a} \left(t_0 \right) = \dot{\vec{v}} \left( t_0 \right) = \ddot{\vec{x}}\left(t_0\right) =\left. \frac{d^2\vec{x}}{dt^2} \right|_{t = t_0}
			\end{equation}
		\end{defn}
		\begin{defn}
			A \textbf{world line} is a curve in $\R \times \R^3$ Galilean space which appears in every Galilean coordinate system as the graph of motion $\vec{x} : \, \R \to \R^3$
		\end{defn}
		Therefore, we may obtain the \textbf{motion of a system of $n$ particles} by first noting that each one has a motion mapping $\vec{x}_i = \R \to \R^3$ for $i \in \{ 1, \, \dots, n\}$ defining their respective world lines. We take these motion mappings to yield a single mapping $\vec{x}: \, \R \to \R^N$ for $N = 3n$ of the time axis into the direct product of $n$ copies of $\R^n$.
		
		\newpage
		\subsection{Newton's Laws of Motion}
		Recall that all motions of a system are uniquely determined by their initial positions $\vec{x} \left(t_0\right) \in \R^N$ and initial velocities $\vec{v} \left(t_0\right) \in \R^N $. Particularly, initial positions and velocities will uniquely determine the motion of a system, where a function $\vec{F}: \, \R^N \times \R^N \times \R \to \R^N$ exists such that
		\begin{equation}
			\label{eq:NewtonEquation}
			\boxed{\ddot{\vec{x}} = \vec{a} = \vec{F} \left(\vec{x}, \dot{\vec{x}}, t \right)}
		\end{equation}
		By Galileo's principle of relativity, there is a selected inertial coordinate system in space-time whose Galilean transform, when applied to all world lines of the points of a mechanical system, yields the world lines of the same system with \textit{new initial conditions}. This fact implies that $\vec{F}\left(\vec{x}, \dot{\vec{x}}, t\right)$ is invariant with respect to the group of Galilean transforms. Moreover, invariance with respect to time indicates that \textit{the laws of physics remain constant}. 
		\begin{exmp} \textit{Time invariance} \\
			Let $\vec{x} = \vec{\phi} (t)$ be a solution to $\ddot{\vec{x}} = \vec{F} \left(\vec{x}, \dot{\vec{x}}, t \right)$. Then, $\forall \, s \in \R, \quad \vec{x} = \vec{\phi} \left(t+s\right)$ is also a solution.
		\end{exmp}
		It follows that inertial coordinate systems will not require time dependence for equation (\ref{eq:NewtonEquation}), implying that $\ddot{\vec{x}} = \vec{\Phi} \left(\vec{x}, \dot{\vec{x}}\right)$
		\begin{exmp} \textit{Spatial translation invariance}\\
			Invariance implies that \textit{space is homogeneous}, meaning that the laws of physics are equivalent for any point chosen in physical space. More explicitly, $ \vec{x}_i = \vec{\phi}_i (t) $ for $ i \in \{1, \, \dots, n\}$ is the motion of some system with $n$ particles satisfying $\ddot{\vec{x}} = \vec{F} \left( \vec{x}, \dot{\vec{x}}, t \right)$. Hence, $\forall \, \vec{r} \in \R^3$ it follows that the motion $\vec{\phi}_i (t) + \vec{r}$ for $i \in \{1, \, \dots, n\}$ also satisfies $\ddot{\vec{x}} = \vec{F} \left( \vec{x}, \dot{\vec{x}}, t \right)$, meaning that the inertial coordinate system must only be dependent on \textit{relative coordinates} $\vec{x}_j - \vec{x}_k$
		\end{exmp}
		Furthermore, if we assume a system under uniform motion such that $\ddot{\vec{x}}_i$ and $\vec{x}_j - \vec{x}_k$ remain unchanged, and add a fixed velocity vector $\vec{v}$ to each $\dot{\vec{x}}_j$, then the inertial coordinate system depends only on \textit{relative velocities} where $\ddot{\vec{x}} = \vec{f} \left( \{\vec{x}_j - \vec{x}_k, \,\dot{\vec{x}}_j - \dot{\vec{x}}_k\}\right)$ for $i,j,k \in \{1, \, \dots, n\}$
		\begin{exmp} \textit{Rotational invariance} \\
			If rotations remain invariant in three dimensions, then it follows that \textit{space is isotropic} such that no direction is preferred. So, if $\vec{\phi}_i : \, \R \to \R^3$ for $i \in \{ 1, \, \dots, n\}$ is a motion of a system of particles satisfying $\ddot{\vec{x}} = \vec{F} \left( \vec{x}, \dot{\vec{x}}, t \right)$, and $G: \, \R^3 \to \R^3$ is an orthogonal transformation, then the motion given by $G \vec{\phi}_i : \, \R \to \R^3$ for $i \in \{1, \, \dots, n\}$ implies that
			$$ \vec{F} \left(G \vec{x}, G \dot{\vec{x}} \right) = G \vec{F} \left(\vec{x}, \dot{\vec{x}} \right)$$ 
		\end{exmp}
		\begin{thm}
			Single particle mechanical systems in an inertial coordinate system will have their acceleration equal to zero.
		\end{thm}
		\begin{proof}
			Assume we have a particle with acceleration $\ddot{\vec{x}}_1 \left(t\right)= \vec{a}_1 \neq \vec{0}$ in some inertial coordinate system, and the same particle is observed with acceleration $\ddot{\vec{x}}_2 \left( t\right) = \vec{a}_2 \neq \vec{0}$ in another inertial coordinate system for time some $t \in \R$. It follows that the difference in velocity of the particle in the first system in relation to its velocity on the second system is given by
			$$ \vec{V} = \vec{v}_1 (t) - \vec{v}_2 (t)  $$
			If an observer sees a particle in an inertial frame, then the observer must also be on an inertial coordinate system by definition. Therefore, $\vec{V}$ is constant and hence not time dependent. Now recall that acceleration is the second derivative of motion with respect to time, implying that
			$$ \frac{d\vec{V}}{dt} = \frac{d \vec{v}_1 (t)}{dt} - \frac{d \vec{v}_2(t)}{dt} $$
			Which directly implies that $\vec{a}_1 = \vec{a}_2$ for $\frac{d\vec{V}}{dt} = 0$ to hold if the coordinate systems are inertial, so the observer would see the particle experiencing no acceleration.
		\end{proof}
		\begin{thm}
			If a mechanical system consists of two particles in some inertial coordinate system where $\dot{\vec{x}}_1 \left(t_0\right) = \dot{\vec{x}}_2 \left(t_0\right) = \vec{0}$, then their mutual distance and direction will remain invariant under Galilean transformations (They remain on the same line).
		\end{thm}
		\begin{proof}
			From time and spatial translation invariance, we can assume that
			$$ \ddot{\vec{x}}_i = \vec{f}_i \left(  \vec{x}_1 - \vec{x}_2, \,\,\dot{\vec{x}}_1 - \dot{\vec{x}}_2\right) \quad \text{for } i \in \{1,2\} $$ 
			If for some $t_0$ it follows that $\left(\vec{x}_1 - \vec{x}_2\right) \cdot \left(\dot{\vec{x}}_1 - \dot{\vec{x}}_2\right) = \left|\left| \vec{x}_1 - \vec{x}_2\right|\right| \cdot \left|\left| \dot{\vec{x}}_1 - \dot{\vec{x}}_2 \right|\right|$, then the rotation Galilean transform $G$ about the vector $\vec{x}_1 - \vec{x}_2$ implies invariance for both vectors. Furthermore, by rotational invariance we have that
			\begin{align*}
				\vec{f}_i \left(\vec{x}_1 - \vec{x}_2, \,\, \dot{\vec{x}}_1 -  \dot{\vec{x}}_2 \right) &= \vec{f}_i \left( G \left(\vec{x}_1 - \vec{x}_2\right), G \left(\dot{\vec{x}}_1 - \dot{\vec{x}}_2 \right)\right) \\
				&= G \vec{f}_i  \left(\vec{x}_1 - \vec{x}_2, \,\, \dot{\vec{x}}_1 -  \dot{\vec{x}}_2  \right) 
			\end{align*}
			So for the same $t_0$ it follows that the accelerations for both particles will also be parallel to $\vec{x}_1 - \vec{x}_2$, meaning the distance and direction between both particles remains invariant under any Galilean transform.
		\end{proof}
	
		\newpage
		
		\begin{exmp}
			\textit{Falling garlic bread} \\
			Assume you spend months planning the launch of your tasty garlic bread into the Earth's stratosphere with a weather balloon. As expected, your garlic bread will reach a height of about $\vec{x}(t)=35,000$ m, where the environmental pressure is significantly lower than at the surface. The lower pressure then will make the balloon burst, allowing the garlic bread to fall back to the surface. From experiments, we know that the garlic bread will experience a constant vertical acceleration of
			$$ \ddot{x} = -9.81 \text{ m}\cdot\text{s}^{-1}$$
			We may now introduce the potential energy of the garlic bread $U=gx$, where $g = 9.81 \text{ m}\cdot\text{s}^{-1}$ such that
			$$ \ddot{x} = - \frac{dU}{dx}$$
			For $N-$dimensional Euclidean spaces $E^N$, we assume that the potential energy $U:\, E^N \to \R$ is differentiable with respect to $E^N$ such that $\vec{\nabla} U = \frac{\partial U}{\partial \vec{x}}$ is the potential gradient vector. Let $E^N = E^{n_1} \times \dots \times E^{n_k}$ be the direct product of euclidean spaces. We denote a point $\vec{x} \in E^N$ by a tuple of $k$ components as $\vec{x} = \left( \vec{x}_1, \, \dots, \vec{x}_k \right)$, and similarly to the potential gradient vector $\vec{\nabla} U = \left( \frac{\partial U}{\partial \vec{x}_1}, \, \dots, \frac{\partial U}{\partial x_k}\right)$. Particularly in the case where $x_1, \, \dots, x_N$ are Cartesian coordinates in $E^N$, the components of the potential gradient vector become $\frac{\partial U}{\partial x_1}, \, \dots, \frac{\partial U}{\partial x_N}$. 
			It is clear by experimentation that the radius vector for our garlic bread with respect to an origin established on the surface of the Earth will satisfy the following equation:
			$$ \ddot{\vec{x}} = - \frac{\partial U}{\partial \vec{x}}$$
			Where $U = -\left(\vec{g}, \vec{x}\right)$. 
			
			It makes sense to now assume the garlic bread goes much further than this making $x >> 35$ km. To more precisely determine its movements, we must assume the origin of our coordinate system lies at the center of the Earth, meaning the vertical vector becomes a radius vector $\vec{r}$, and we assume a constant distance to the Earth $r_0$ such that $\left|\left| \vec{r} \right|\right| = r_0 + x$. We can now define the gravitational acceleration with an inverse square law of the radius
			$$ \ddot{x} = -g \frac{r_0^2}{\left(r_0 + x\right)^2}$$
			Now, considering the potential energy $U = -\frac{g r_0^2}{r_0 + x}$, this yields
			$$ \ddot{x} = \frac{U}{r_0 + x}$$ 
		\end{exmp}
		\begin{exe}
			Determine the escape velocity $v_{esc}$ of the garlic bread. Recall that the escape velocity is the minimal speed for an object to escape the gravitational potential of the Earth such that it does not return to the surface. (\textit{Hint: Integrals are your friends.})
		\end{exe}
		We end this chapter with the introduction of \textbf{conservative systems}, where the difference in total energy between any two events is always zero. Let $E^{3n} = E^3 \times \dots \times E^3$ be the configuration space of a system of $n$ particles in three-dimensional Euclidean space $E^3$. Now suppose the potential energy $U: \, E^{3n} \to \R$ is a differentiable function, and let $m_1, \dots, m_n$ be positive real numbers representing the mass of each particle. The motion of $n$ particles of respective masses $m_1, \dots, m_n$ in the potential field with potential energy $U$ is given by the second degree ordinary differential equation
		\begin{equation}
			\label{eq:MotionMass}
			\boxed{m_i \ddot{\vec{x}}_i = \frac{\partial U}{\partial \vec{x}_i}, \quad i \in \{1, \dots, n\}}
		\end{equation}
		\newpage
		
		\section{Equations of Motion}
		\subsection{Constraints}
		To continue, we must define a few useful concepts for the study of the equations of motion.
		\begin{defn}
			A system with \textbf{one degree of freedom} is described by a single second-order differential equation
			\begin{equation}
				\label{eq:OneDegree}
				\boxed{\ddot{x} = f(x), \quad x \in \R}
			\end{equation}
		\end{defn}
		\begin{defn}
			The \textbf{kinetic energy} of a system with one degree of freedom is
			\begin{equation}
				\label{eq:KineticOneDegree}
				\boxed{T\left(\dot{x}\right) = \frac12 \dot{x}^2}
			\end{equation}
		\end{defn}
		\begin{defn}
			The \textbf{potential energy} of a system with one degree of freedom is
			\begin{equation}
				\label{eq:PotentialOneDegree}
				\boxed{U(x) = -\int_{x_0}^x f(u)du}
			\end{equation}
		\end{defn}
		\begin{defn}
			The \textbf{total energy} of a system is 
			\begin{equation}
				\label{eq:TotalEnergy}
				\boxed{E \left(x, \dot{x}\right) = T\left(\dot{x}\right)+U(x)}
			\end{equation}
		\end{defn}
		From these definitions, it is clear that the potential energy determines the function that represents $\ddot{x}$.
		\begin{thm}
			The law of conservation of energy implies that the total energy of particles experiencing motion in a conservative system will not vary with respect to time. 
		\end{thm}
		\begin{proof}
			\begin{align*}
				\frac{d}{dt} \left( T\left(\dot{x} \right) + U(x)\right) &= \dot{x}\ddot{x} + \frac{dU}{dx} \dot{x}\\ 
				&= \dot{x} \left( \ddot{x} - f(x)\right)
			\end{align*}
		\end{proof}
		\subsection{Angular Momentum}
		\subsection{Systems of Particles}
\end{document}
